% Options for packages loaded elsewhere
\PassOptionsToPackage{unicode}{hyperref}
\PassOptionsToPackage{hyphens}{url}
%
\documentclass[
]{article}
\usepackage{amsmath,amssymb}
\usepackage{lmodern}
\usepackage{iftex}
\ifPDFTeX
  \usepackage[T1]{fontenc}
  \usepackage[utf8]{inputenc}
  \usepackage{textcomp} % provide euro and other symbols
\else % if luatex or xetex
  \usepackage{unicode-math}
  \defaultfontfeatures{Scale=MatchLowercase}
  \defaultfontfeatures[\rmfamily]{Ligatures=TeX,Scale=1}
\fi
% Use upquote if available, for straight quotes in verbatim environments
\IfFileExists{upquote.sty}{\usepackage{upquote}}{}
\IfFileExists{microtype.sty}{% use microtype if available
  \usepackage[]{microtype}
  \UseMicrotypeSet[protrusion]{basicmath} % disable protrusion for tt fonts
}{}
\makeatletter
\@ifundefined{KOMAClassName}{% if non-KOMA class
  \IfFileExists{parskip.sty}{%
    \usepackage{parskip}
  }{% else
    \setlength{\parindent}{0pt}
    \setlength{\parskip}{6pt plus 2pt minus 1pt}}
}{% if KOMA class
  \KOMAoptions{parskip=half}}
\makeatother
\usepackage{xcolor}
\usepackage[margin=1in]{geometry}
\usepackage{color}
\usepackage{fancyvrb}
\newcommand{\VerbBar}{|}
\newcommand{\VERB}{\Verb[commandchars=\\\{\}]}
\DefineVerbatimEnvironment{Highlighting}{Verbatim}{commandchars=\\\{\}}
% Add ',fontsize=\small' for more characters per line
\usepackage{framed}
\definecolor{shadecolor}{RGB}{248,248,248}
\newenvironment{Shaded}{\begin{snugshade}}{\end{snugshade}}
\newcommand{\AlertTok}[1]{\textcolor[rgb]{0.94,0.16,0.16}{#1}}
\newcommand{\AnnotationTok}[1]{\textcolor[rgb]{0.56,0.35,0.01}{\textbf{\textit{#1}}}}
\newcommand{\AttributeTok}[1]{\textcolor[rgb]{0.77,0.63,0.00}{#1}}
\newcommand{\BaseNTok}[1]{\textcolor[rgb]{0.00,0.00,0.81}{#1}}
\newcommand{\BuiltInTok}[1]{#1}
\newcommand{\CharTok}[1]{\textcolor[rgb]{0.31,0.60,0.02}{#1}}
\newcommand{\CommentTok}[1]{\textcolor[rgb]{0.56,0.35,0.01}{\textit{#1}}}
\newcommand{\CommentVarTok}[1]{\textcolor[rgb]{0.56,0.35,0.01}{\textbf{\textit{#1}}}}
\newcommand{\ConstantTok}[1]{\textcolor[rgb]{0.00,0.00,0.00}{#1}}
\newcommand{\ControlFlowTok}[1]{\textcolor[rgb]{0.13,0.29,0.53}{\textbf{#1}}}
\newcommand{\DataTypeTok}[1]{\textcolor[rgb]{0.13,0.29,0.53}{#1}}
\newcommand{\DecValTok}[1]{\textcolor[rgb]{0.00,0.00,0.81}{#1}}
\newcommand{\DocumentationTok}[1]{\textcolor[rgb]{0.56,0.35,0.01}{\textbf{\textit{#1}}}}
\newcommand{\ErrorTok}[1]{\textcolor[rgb]{0.64,0.00,0.00}{\textbf{#1}}}
\newcommand{\ExtensionTok}[1]{#1}
\newcommand{\FloatTok}[1]{\textcolor[rgb]{0.00,0.00,0.81}{#1}}
\newcommand{\FunctionTok}[1]{\textcolor[rgb]{0.00,0.00,0.00}{#1}}
\newcommand{\ImportTok}[1]{#1}
\newcommand{\InformationTok}[1]{\textcolor[rgb]{0.56,0.35,0.01}{\textbf{\textit{#1}}}}
\newcommand{\KeywordTok}[1]{\textcolor[rgb]{0.13,0.29,0.53}{\textbf{#1}}}
\newcommand{\NormalTok}[1]{#1}
\newcommand{\OperatorTok}[1]{\textcolor[rgb]{0.81,0.36,0.00}{\textbf{#1}}}
\newcommand{\OtherTok}[1]{\textcolor[rgb]{0.56,0.35,0.01}{#1}}
\newcommand{\PreprocessorTok}[1]{\textcolor[rgb]{0.56,0.35,0.01}{\textit{#1}}}
\newcommand{\RegionMarkerTok}[1]{#1}
\newcommand{\SpecialCharTok}[1]{\textcolor[rgb]{0.00,0.00,0.00}{#1}}
\newcommand{\SpecialStringTok}[1]{\textcolor[rgb]{0.31,0.60,0.02}{#1}}
\newcommand{\StringTok}[1]{\textcolor[rgb]{0.31,0.60,0.02}{#1}}
\newcommand{\VariableTok}[1]{\textcolor[rgb]{0.00,0.00,0.00}{#1}}
\newcommand{\VerbatimStringTok}[1]{\textcolor[rgb]{0.31,0.60,0.02}{#1}}
\newcommand{\WarningTok}[1]{\textcolor[rgb]{0.56,0.35,0.01}{\textbf{\textit{#1}}}}
\usepackage{graphicx}
\makeatletter
\def\maxwidth{\ifdim\Gin@nat@width>\linewidth\linewidth\else\Gin@nat@width\fi}
\def\maxheight{\ifdim\Gin@nat@height>\textheight\textheight\else\Gin@nat@height\fi}
\makeatother
% Scale images if necessary, so that they will not overflow the page
% margins by default, and it is still possible to overwrite the defaults
% using explicit options in \includegraphics[width, height, ...]{}
\setkeys{Gin}{width=\maxwidth,height=\maxheight,keepaspectratio}
% Set default figure placement to htbp
\makeatletter
\def\fps@figure{htbp}
\makeatother
\setlength{\emergencystretch}{3em} % prevent overfull lines
\providecommand{\tightlist}{%
  \setlength{\itemsep}{0pt}\setlength{\parskip}{0pt}}
\setcounter{secnumdepth}{-\maxdimen} % remove section numbering
\usepackage{caption}
\ifLuaTeX
  \usepackage{selnolig}  % disable illegal ligatures
\fi
\IfFileExists{bookmark.sty}{\usepackage{bookmark}}{\usepackage{hyperref}}
\IfFileExists{xurl.sty}{\usepackage{xurl}}{} % add URL line breaks if available
\urlstyle{same} % disable monospaced font for URLs
\hypersetup{
  pdftitle={ECOL 610: NEON Assignment 1},
  pdfauthor={Group - Santa Rita Experimental Range},
  hidelinks,
  pdfcreator={LaTeX via pandoc}}

\title{ECOL 610: NEON Assignment 1}
\author{Group - Santa Rita Experimental Range}
\date{13 September, 2022}

\begin{document}
\maketitle

\begin{itemize}
\tightlist
\item
  Emily Swartz
\item
  Shahriar Shah Heydari
\item
  Stephanie Cardinalli
\item
  George Woolsey
\end{itemize}

\hypertarget{in-class}{%
\section{In Class}\label{in-class}}

\hypertarget{setup}{%
\subsection{Setup}\label{setup}}

This is the R script for ECOL610 NEON assignment 1 This script will
provide code to carry out assignment 1 for the Central Plains
Experiemntal Range - you will need to modify this code to carry it out
for your site.

First, load in the needed packages. Install the packages if needed.

\begin{Shaded}
\begin{Highlighting}[]
\FunctionTok{library}\NormalTok{(tidyverse)}
\FunctionTok{library}\NormalTok{(lubridate)}
\FunctionTok{library}\NormalTok{(viridis)}
\FunctionTok{library}\NormalTok{(scales)}
\FunctionTok{library}\NormalTok{(latex2exp)}
\end{Highlighting}
\end{Shaded}

Load in the data. You will need to change the file path for your
directory.

\begin{Shaded}
\begin{Highlighting}[]
\CommentTok{\#load in the data}
\NormalTok{CPER\_30 }\OtherTok{\textless{}{-}} \FunctionTok{read.csv}\NormalTok{(}\StringTok{"../data/Central Plains Experimental Range {-} 30 min.csv"}\NormalTok{)}
\NormalTok{CPER\_daily }\OtherTok{\textless{}{-}} \FunctionTok{read.csv}\NormalTok{(}\StringTok{"../data/Central Plains Experimental Range {-} daily.csv"}\NormalTok{)}
\end{Highlighting}
\end{Shaded}

\hypertarget{question-1}{%
\subsection{Question 1}\label{question-1}}

Calculate GPP and compare to NEON GPP for summer and winter week

\emph{NEE is defined, by convention, as CO2 flux from the ecosystem to
the atmosphere. It corresponds to a negative carbon input to ecosystems.
NEE is defined in this way because atmospheric scientists, who
originated the term, seek to document net sources of CO2 to the
atmosphere (i.e., NEE) that account for rising atmospheric CO2
concentration. Therefore, CO2 input to the ecosystem is a negative NEE.}

\href{https://d1wqtxts1xzle7.cloudfront.net/55690956/Principles_of_terrestrial_ecosystem_ecology-with-cover-page-v2.pdf?Expires=1663106506\&Signature=ZLKRpouXVl6Q2oVAvMbYfcyWZT227z~A0BOTNMvx3nr-hzPv-aQr2DF-vvK~O2T8ygmVtbYXdNlXfNAE8FYZ70B2OOHPU8HHIhXPwKW90Mf~SYyj2xIQBIb0gMK4mZ6lJLG~eO7cPoLuK974yvVy5zdcnJt81MhsSB2vPb3w8l-QijHyNlYmpr43FYR50UuYNAib58kuaUNYxN-jMFaLVLS6fvYxV93ToeH3mILBD3mMliAUAViXzXngzVVuLQXXyJodsR1JbR54PJ-Uhyeitj7PI9Qq1Rtpz1Y0gRIkXd5DiJenTOXLTpc1jD~OYBqyGowRjcSPMSgyMT1cilxGQQ__\&Key-Pair-Id=APKAJLOHF5GGSLRBV4ZA}{Chapin,
F. S., Matson, P. A., Mooney, H. A., \& Vitousek, P. M. (2002).
Principles of terrestrial ecosystem ecology.} p.208

\$\$

\begin{aligned}
NEE = R_{E} - GPP

\end{aligned}

\[
\]

\begin{aligned}
GPP = R_{E} - NEE
\end{aligned}

\$\$

\begin{Shaded}
\begin{Highlighting}[]
\CommentTok{\#calculate GPP from NEE and Re using equation NEE = Re {-} GPP (negative values land C storage)}
\NormalTok{CPER\_30 }\OtherTok{\textless{}{-}}\NormalTok{ CPER\_30 }\SpecialCharTok{\%\textgreater{}\%} 
\NormalTok{  dplyr}\SpecialCharTok{::}\FunctionTok{mutate}\NormalTok{(}
    \AttributeTok{GPP\_calc =} \FunctionTok{as.numeric}\NormalTok{(RE }\SpecialCharTok{{-}}\NormalTok{ NEE)}
\NormalTok{    , }\AttributeTok{GPP\_is\_equal =} \FunctionTok{round}\NormalTok{(GPP\_calc, }\DecValTok{3}\NormalTok{) }\SpecialCharTok{==} \FunctionTok{round}\NormalTok{(GPP, }\DecValTok{3}\NormalTok{)}
\NormalTok{  )}

\CommentTok{\# are there rows where the data is not equal?}
  \FunctionTok{nrow}\NormalTok{(}
\NormalTok{    CPER\_30 }\SpecialCharTok{\%\textgreater{}\%} 
\NormalTok{      dplyr}\SpecialCharTok{::}\FunctionTok{filter}\NormalTok{(}
\NormalTok{        GPP\_is\_equal }\SpecialCharTok{==} \ConstantTok{FALSE}
        \SpecialCharTok{\&} \SpecialCharTok{!}\FunctionTok{is.na}\NormalTok{(GPP)}
\NormalTok{      )}
\NormalTok{  )}
\end{Highlighting}
\end{Shaded}

\begin{verbatim}
## [1] 8499
\end{verbatim}

\begin{Shaded}
\begin{Highlighting}[]
\CommentTok{\#clipping the data to a winter and summer week}
 \CommentTok{\#choose your own weeks (without data gaps) for your site!}
\CommentTok{\#February 16{-}22, 2020}
\NormalTok{CPER\_winter\_week }\OtherTok{\textless{}{-}} \FunctionTok{filter}\NormalTok{(CPER\_30, Year }\SpecialCharTok{==} \StringTok{"2020"} \SpecialCharTok{\&}\NormalTok{ DOY }\SpecialCharTok{==} \DecValTok{47}\SpecialCharTok{:}\DecValTok{53}\NormalTok{)}
\CommentTok{\#June 21{-}27, 2020}
\NormalTok{CPER\_summer\_week }\OtherTok{\textless{}{-}} \FunctionTok{filter}\NormalTok{(CPER\_30, Year }\SpecialCharTok{==} \StringTok{"2020"} \SpecialCharTok{\&}\NormalTok{ DOY }\SpecialCharTok{==} \DecValTok{173}\SpecialCharTok{:}\DecValTok{180}\NormalTok{)}

\CommentTok{\#creating long format for plotting}
\NormalTok{CWW\_stacked }\OtherTok{\textless{}{-}}\FunctionTok{gather}\NormalTok{(CPER\_winter\_week, }\StringTok{"GPP\_type"}\NormalTok{, }\StringTok{"GPP\_value"}\NormalTok{, }\FunctionTok{c}\NormalTok{(}\DecValTok{9}\NormalTok{,}\DecValTok{15}\NormalTok{))}
\NormalTok{CSW\_stacked }\OtherTok{\textless{}{-}}\FunctionTok{gather}\NormalTok{(CPER\_summer\_week, }\StringTok{"GPP\_type"}\NormalTok{, }\StringTok{"GPP\_value"}\NormalTok{, }\FunctionTok{c}\NormalTok{(}\DecValTok{9}\NormalTok{,}\DecValTok{15}\NormalTok{))}

\CommentTok{\#winter}
\CommentTok{\#plot of NEON vs calculated GPP}
\FunctionTok{ggplot}\NormalTok{(CWW\_stacked,}\FunctionTok{aes}\NormalTok{(}\AttributeTok{x=}\NormalTok{DOY.total,}\AttributeTok{y=}\NormalTok{GPP\_value, }\AttributeTok{color =}\NormalTok{ GPP\_type)) }\SpecialCharTok{+} \FunctionTok{geom\_point}\NormalTok{() }\SpecialCharTok{+} 
    \FunctionTok{xlab}\NormalTok{(}\StringTok{"Day of Year"}\NormalTok{) }\SpecialCharTok{+} 
    \FunctionTok{ylab}\NormalTok{(}\FunctionTok{expression}\NormalTok{(}\FunctionTok{paste}\NormalTok{(}\StringTok{"GPP (mol CO"}\NormalTok{[}\DecValTok{2}\NormalTok{]}\SpecialCharTok{*}\StringTok{" m"}\SpecialCharTok{\^{}}\NormalTok{\{}\SpecialCharTok{{-}}\DecValTok{2}\NormalTok{\}, }\StringTok{"day"}\SpecialCharTok{\^{}}\NormalTok{\{}\SpecialCharTok{{-}}\DecValTok{2}\NormalTok{\}}\SpecialCharTok{*}\StringTok{")"}\NormalTok{))) }\SpecialCharTok{+}
    \FunctionTok{theme\_bw}\NormalTok{(}\AttributeTok{base\_size =} \DecValTok{16}\NormalTok{) }\SpecialCharTok{+} \FunctionTok{theme}\NormalTok{(}\AttributeTok{panel.grid.major =} \FunctionTok{element\_blank}\NormalTok{(), }\AttributeTok{panel.grid.minor =} \FunctionTok{element\_blank}\NormalTok{()) }\SpecialCharTok{+} \FunctionTok{ggtitle}\NormalTok{(}\StringTok{"Winter GPP Comparison"}\NormalTok{)}
\end{Highlighting}
\end{Shaded}

\includegraphics{C:/Data/ECOL610/ecol610_neon_assignment1/src/../ProbLab_3_England_files/figure-latex/unnamed-chunk-3-1.pdf}

\begin{Shaded}
\begin{Highlighting}[]
\CommentTok{\#summer}
\CommentTok{\#plot of NEON vs calculated GPP}
\FunctionTok{ggplot}\NormalTok{(CSW\_stacked,}\FunctionTok{aes}\NormalTok{(}\AttributeTok{x=}\NormalTok{DOY.total,}\AttributeTok{y=}\NormalTok{GPP\_value, }\AttributeTok{color =}\NormalTok{ GPP\_type)) }\SpecialCharTok{+} \FunctionTok{geom\_point}\NormalTok{() }\SpecialCharTok{+} 
    \FunctionTok{xlab}\NormalTok{(}\StringTok{"Day of Year"}\NormalTok{) }\SpecialCharTok{+} 
    \FunctionTok{ylab}\NormalTok{(latex2exp}\SpecialCharTok{::}\FunctionTok{TeX}\NormalTok{(}\StringTok{"$GPP }\SpecialCharTok{\textbackslash{}\textbackslash{}}\StringTok{; (mol }\SpecialCharTok{\textbackslash{}\textbackslash{}}\StringTok{; CO\_\{2\} }\SpecialCharTok{\textbackslash{}\textbackslash{}}\StringTok{cdot m\^{}\{{-}2\} }\SpecialCharTok{\textbackslash{}\textbackslash{}}\StringTok{cdot day\^{}\{{-}2\}$"}\NormalTok{)) }\SpecialCharTok{+}
    \FunctionTok{theme\_bw}\NormalTok{(}\AttributeTok{base\_size =} \DecValTok{16}\NormalTok{) }\SpecialCharTok{+} \FunctionTok{theme}\NormalTok{(}\AttributeTok{panel.grid.major =} \FunctionTok{element\_blank}\NormalTok{(), }\AttributeTok{panel.grid.minor =} \FunctionTok{element\_blank}\NormalTok{()) }\SpecialCharTok{+} \FunctionTok{ggtitle}\NormalTok{(}\StringTok{"Summer GPP Comparison"}\NormalTok{)}
\end{Highlighting}
\end{Shaded}

\includegraphics{C:/Data/ECOL610/ecol610_neon_assignment1/src/../ProbLab_3_England_files/figure-latex/unnamed-chunk-3-2.pdf}

\hypertarget{question-2}{%
\subsection{Question 2}\label{question-2}}

annual patterns in C exchange and environmental properties

\begin{Shaded}
\begin{Highlighting}[]
\CommentTok{\#reduce data to just 2020}
\NormalTok{CPER\_2020 }\OtherTok{\textless{}{-}} \FunctionTok{filter}\NormalTok{(CPER\_daily, Year }\SpecialCharTok{==} \StringTok{"2020"}\NormalTok{)}
\CommentTok{\#plot of NEE}
\FunctionTok{ggplot}\NormalTok{(CPER\_2020,}\FunctionTok{aes}\NormalTok{(}\AttributeTok{x=}\NormalTok{DOY,}\AttributeTok{y=}\NormalTok{NEE)) }\SpecialCharTok{+} \FunctionTok{geom\_point}\NormalTok{() }\SpecialCharTok{+} 
    \FunctionTok{xlab}\NormalTok{(}\StringTok{"Day of Year"}\NormalTok{) }\SpecialCharTok{+} 
    \FunctionTok{ylab}\NormalTok{(latex2exp}\SpecialCharTok{::}\FunctionTok{TeX}\NormalTok{(}\StringTok{"$NEE }\SpecialCharTok{\textbackslash{}\textbackslash{}}\StringTok{; (mol }\SpecialCharTok{\textbackslash{}\textbackslash{}}\StringTok{; CO\_\{2\} }\SpecialCharTok{\textbackslash{}\textbackslash{}}\StringTok{cdot m\^{}\{{-}2\} }\SpecialCharTok{\textbackslash{}\textbackslash{}}\StringTok{cdot day\^{}\{{-}2\}$"}\NormalTok{)) }\SpecialCharTok{+}
    \FunctionTok{theme\_bw}\NormalTok{(}\AttributeTok{base\_size =} \DecValTok{16}\NormalTok{) }\SpecialCharTok{+} \FunctionTok{theme}\NormalTok{(}\AttributeTok{panel.grid.major =} \FunctionTok{element\_blank}\NormalTok{(), }\AttributeTok{panel.grid.minor =} \FunctionTok{element\_blank}\NormalTok{()) }\SpecialCharTok{+} \FunctionTok{ggtitle}\NormalTok{(}\StringTok{"Annual NEE"}\NormalTok{)}
\end{Highlighting}
\end{Shaded}

\includegraphics{C:/Data/ECOL610/ecol610_neon_assignment1/src/../ProbLab_3_England_files/figure-latex/unnamed-chunk-4-1.pdf}

\begin{Shaded}
\begin{Highlighting}[]
\CommentTok{\#plot of GPP}
\FunctionTok{ggplot}\NormalTok{(CPER\_2020,}\FunctionTok{aes}\NormalTok{(}\AttributeTok{x=}\NormalTok{DOY,}\AttributeTok{y=}\NormalTok{GPP)) }\SpecialCharTok{+} \FunctionTok{geom\_point}\NormalTok{() }\SpecialCharTok{+} 
    \FunctionTok{xlab}\NormalTok{(}\StringTok{"Day of Year"}\NormalTok{) }\SpecialCharTok{+} 
    \FunctionTok{ylab}\NormalTok{(latex2exp}\SpecialCharTok{::}\FunctionTok{TeX}\NormalTok{(}\StringTok{"$GPP }\SpecialCharTok{\textbackslash{}\textbackslash{}}\StringTok{; (mol }\SpecialCharTok{\textbackslash{}\textbackslash{}}\StringTok{; CO\_\{2\} }\SpecialCharTok{\textbackslash{}\textbackslash{}}\StringTok{cdot m\^{}\{{-}2\} }\SpecialCharTok{\textbackslash{}\textbackslash{}}\StringTok{cdot day\^{}\{{-}2\}$"}\NormalTok{)) }\SpecialCharTok{+}
    \FunctionTok{theme\_bw}\NormalTok{(}\AttributeTok{base\_size =} \DecValTok{16}\NormalTok{) }\SpecialCharTok{+} \FunctionTok{theme}\NormalTok{(}\AttributeTok{panel.grid.major =} \FunctionTok{element\_blank}\NormalTok{(), }\AttributeTok{panel.grid.minor =} \FunctionTok{element\_blank}\NormalTok{()) }\SpecialCharTok{+} \FunctionTok{ggtitle}\NormalTok{(}\StringTok{"Annual GPP"}\NormalTok{)}
\end{Highlighting}
\end{Shaded}

\includegraphics{C:/Data/ECOL610/ecol610_neon_assignment1/src/../ProbLab_3_England_files/figure-latex/unnamed-chunk-4-2.pdf}

\begin{Shaded}
\begin{Highlighting}[]
\CommentTok{\#plot of Re}
\FunctionTok{ggplot}\NormalTok{(CPER\_2020,}\FunctionTok{aes}\NormalTok{(}\AttributeTok{x=}\NormalTok{DOY,}\AttributeTok{y=}\NormalTok{RE)) }\SpecialCharTok{+} \FunctionTok{geom\_point}\NormalTok{() }\SpecialCharTok{+} 
    \FunctionTok{xlab}\NormalTok{(}\StringTok{"Day of Year"}\NormalTok{) }\SpecialCharTok{+} 
    \FunctionTok{ylab}\NormalTok{(latex2exp}\SpecialCharTok{::}\FunctionTok{TeX}\NormalTok{(}\StringTok{"$R\_\{E\} }\SpecialCharTok{\textbackslash{}\textbackslash{}}\StringTok{; (mol }\SpecialCharTok{\textbackslash{}\textbackslash{}}\StringTok{; CO\_\{2\} }\SpecialCharTok{\textbackslash{}\textbackslash{}}\StringTok{cdot m\^{}\{{-}2\} }\SpecialCharTok{\textbackslash{}\textbackslash{}}\StringTok{cdot day\^{}\{{-}2\}$"}\NormalTok{)) }\SpecialCharTok{+}
    \FunctionTok{theme\_bw}\NormalTok{(}\AttributeTok{base\_size =} \DecValTok{16}\NormalTok{) }\SpecialCharTok{+} \FunctionTok{theme}\NormalTok{(}\AttributeTok{panel.grid.major =} \FunctionTok{element\_blank}\NormalTok{(), }\AttributeTok{panel.grid.minor =} \FunctionTok{element\_blank}\NormalTok{()) }\SpecialCharTok{+} \FunctionTok{ggtitle}\NormalTok{(latex2exp}\SpecialCharTok{::}\FunctionTok{TeX}\NormalTok{(}\StringTok{"$Annual }\SpecialCharTok{\textbackslash{}\textbackslash{}}\StringTok{; R\_\{E\}$"}\NormalTok{))}
\end{Highlighting}
\end{Shaded}

\includegraphics{C:/Data/ECOL610/ecol610_neon_assignment1/src/../ProbLab_3_England_files/figure-latex/unnamed-chunk-4-3.pdf}

\begin{Shaded}
\begin{Highlighting}[]
\CommentTok{\#plot of soil temp}
\FunctionTok{ggplot}\NormalTok{(CPER\_2020,}\FunctionTok{aes}\NormalTok{(}\AttributeTok{x=}\NormalTok{DOY,}\AttributeTok{y=}\NormalTok{TS)) }\SpecialCharTok{+} \FunctionTok{geom\_point}\NormalTok{() }\SpecialCharTok{+} 
    \FunctionTok{xlab}\NormalTok{(}\StringTok{"Day of Year"}\NormalTok{) }\SpecialCharTok{+} 
    \FunctionTok{ylab}\NormalTok{(}\StringTok{"Soil temperature (\textbackslash{}u00B0C)"}\NormalTok{) }\SpecialCharTok{+}
    \FunctionTok{theme\_bw}\NormalTok{(}\AttributeTok{base\_size =} \DecValTok{16}\NormalTok{) }\SpecialCharTok{+} \FunctionTok{theme}\NormalTok{(}\AttributeTok{panel.grid.major =} \FunctionTok{element\_blank}\NormalTok{(), }\AttributeTok{panel.grid.minor =} \FunctionTok{element\_blank}\NormalTok{()) }\SpecialCharTok{+} \FunctionTok{ggtitle}\NormalTok{(}\StringTok{"Annual Soil Temperature"}\NormalTok{)}
\end{Highlighting}
\end{Shaded}

\includegraphics{C:/Data/ECOL610/ecol610_neon_assignment1/src/../ProbLab_3_England_files/figure-latex/unnamed-chunk-4-4.pdf}

\begin{Shaded}
\begin{Highlighting}[]
\CommentTok{\#plot of air temp}
\FunctionTok{ggplot}\NormalTok{(CPER\_2020,}\FunctionTok{aes}\NormalTok{(}\AttributeTok{x=}\NormalTok{DOY,}\AttributeTok{y=}\NormalTok{TA)) }\SpecialCharTok{+} \FunctionTok{geom\_point}\NormalTok{() }\SpecialCharTok{+} 
    \FunctionTok{xlab}\NormalTok{(}\StringTok{"Day of Year"}\NormalTok{) }\SpecialCharTok{+} 
    \FunctionTok{ylab}\NormalTok{(}\StringTok{"Air temperature (\textbackslash{}u00B0C)"}\NormalTok{) }\SpecialCharTok{+}
    \FunctionTok{theme\_bw}\NormalTok{(}\AttributeTok{base\_size =} \DecValTok{16}\NormalTok{) }\SpecialCharTok{+} \FunctionTok{theme}\NormalTok{(}\AttributeTok{panel.grid.major =} \FunctionTok{element\_blank}\NormalTok{(), }\AttributeTok{panel.grid.minor =} \FunctionTok{element\_blank}\NormalTok{()) }\SpecialCharTok{+} \FunctionTok{ggtitle}\NormalTok{(}\StringTok{"Annual Air Temperature"}\NormalTok{)}
\end{Highlighting}
\end{Shaded}

\includegraphics{C:/Data/ECOL610/ecol610_neon_assignment1/src/../ProbLab_3_England_files/figure-latex/unnamed-chunk-4-5.pdf}

\begin{Shaded}
\begin{Highlighting}[]
\CommentTok{\#plot of soil moisture}
\FunctionTok{ggplot}\NormalTok{(CPER\_2020,}\FunctionTok{aes}\NormalTok{(}\AttributeTok{x=}\NormalTok{DOY,}\AttributeTok{y=}\NormalTok{SWC)) }\SpecialCharTok{+} \FunctionTok{geom\_point}\NormalTok{() }\SpecialCharTok{+} 
    \FunctionTok{xlab}\NormalTok{(}\StringTok{"Day of Year"}\NormalTok{) }\SpecialCharTok{+} 
    \FunctionTok{ylab}\NormalTok{(}\StringTok{"Soil Water Content (\%)"}\NormalTok{) }\SpecialCharTok{+}
    \FunctionTok{theme\_bw}\NormalTok{(}\AttributeTok{base\_size =} \DecValTok{16}\NormalTok{) }\SpecialCharTok{+} \FunctionTok{theme}\NormalTok{(}\AttributeTok{panel.grid.major =} \FunctionTok{element\_blank}\NormalTok{(), }\AttributeTok{panel.grid.minor =} \FunctionTok{element\_blank}\NormalTok{()) }\SpecialCharTok{+} \FunctionTok{ggtitle}\NormalTok{(}\StringTok{"Annual Soil Water Content"}\NormalTok{)}
\end{Highlighting}
\end{Shaded}

\includegraphics{C:/Data/ECOL610/ecol610_neon_assignment1/src/../ProbLab_3_England_files/figure-latex/unnamed-chunk-4-6.pdf}

Note that the \texttt{echo\ =\ FALSE} parameter was added to the code
chunk to prevent printing of the R code that generated the plot.

\hypertarget{assignment}{%
\section{Assignment}\label{assignment}}

The draft questions for that assignment are:

\hypertarget{question-1-1}{%
\subsection{Question 1}\label{question-1-1}}

Using the 30 minute data. Calculate GPP for your site. Choose a winter
and summer week in your dataset. Create a plot with both your calculated
GPP and the NEON GPP.

\hypertarget{a.}{%
\subsubsection{a.}\label{a.}}

How do the calculated and NEON GPP's compare for your site? Why are they
the same or different? (Note: NEON uses eddy covariance to calculate
these metrics - more about that here (Links to an external site.).)

\hypertarget{b.}{%
\subsubsection{b.}\label{b.}}

How do your sites GPP values vary between summer and winter? What do you
think is driving these differences?

\hypertarget{c.}{%
\subsubsection{c.~}\label{c.}}

CPER is a semi-arid grassland. Compare GPP values between your site and
CPER - why might they be different or similar?

\hypertarget{question-2-1}{%
\subsection{Question 2}\label{question-2-1}}

Using the daily data. Select a single year of data for your site. Plot
NEE, GPP, Re, soil temperature, air temperature, and soil water content
against Day of Year.

\hypertarget{a.-1}{%
\subsubsection{a.}\label{a.-1}}

Describe the annual patterns in each plot and what you think drives
them.

\hypertarget{b.-1}{%
\subsubsection{b.}\label{b.-1}}

Compare annual values of NEE at your site vs.~CPER. Which exchanges more
carbon? Why do you think that is?

\end{document}
